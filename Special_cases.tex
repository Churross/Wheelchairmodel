\documentclass[a4paper,titlepage,10pt]{article}
\usepackage[utf8]{inputenc}
\usepackage{amsmath}
\usepackage{amsfonts}
\usepackage{amssymb}
\usepackage{graphicx}
\usepackage{a4wide}

\title{Logbook \\ DBL 2IO70}

\author{Author: N.M.J. Ras \\
                Group 20 }
\date{03-03-2015}

\begin{document}
\maketitle
\section{Special cases}
\subsection{Employees and maintainers}
The salary is in euro's/hour. This means the more hours are being worked, the more the cost for the KLM will be. Except for when the people work for free. So that they are volunteers.\\
\begin{equation}
CostEmployed = Salary * NrEmployed
\end{equation}
If the extremes are evaluated, then taking the limit:
\begin{equation}
\lim_{Salary \rightarrow 0}CostEmployed = Salary * NrEmployed = \lim_{NrEmployed \rightarrow 0}CostEmployed = Salary * NrEmployed = 0
\end{equation}
So by either having no escorts or maintainers and/or no salary, there will be no costs. However this also results in no service hence this situation is not desired. On the other hand taking the limit to infinity;
\begin{equation}
\lim_{Salary \rightarrow \infty} CostEmployed = Salary * NrEmployed = \lim_{NrEmployed \rightarrow 0} CostEmployed = Salary * NrEmployed = \infty
\end{equation}
This is neither desired as result for the costs. Therefore neither the salary or the number of employed should be infinite or 0.
\begin{equation}
\lim_{WheelChairsPerMaintainer \rightarrow 0} NrMaintainers = \frac{NrWheelchairs}{WheelchairsPerMaintainer}
\end{equation}
\begin{equation}
= \lim_{NrMaintainers \rightarrow 0}NrMaintainers = \frac{NrWheelchairs}{WheelchairPerMaintainer} = 0
\end{equation}


\subsubsection{Employees are free}
In this case, escorts would be volunteers. People who like to work with disabled people could apply for the job and help the KLM by bringing disabled people from their plane to the correct other plane. If this would happen, more people could be hired as volunteer and this also means that people could stay and wait together with the disabled people. Do some games take them to a diner or anything else. It is a real thing to think about, because people sometimes like to work with those people. On the other hand those people do not have the same responsibilities as someone who is getting paid for the job. This could be a disadvantage, but if you have double the amount of people and they are willing to work kind of hard then this is definitely a good opportunity.
\subsubsection{No escorts}
If there are no escorts, no people can be brought to their gate. This means either no plane will leave in time or there are by coincidence no disabled people flying that day.
\subsubsection{Infinite escorts}
If there are infinite many escorts, that means it will cost the KLM lots of money, except if all those escorts are volunteers or if just a part of the escorts are volunteers. In either case probably every plane will leave in time, because if there are so many escorts, they can bring all the disabled persons to their gate.
\subsubsection{No wheelchair maintainers}
If there are no wheelchair maintainers, no wheelchair can be fixed and thus everytime a wheelchair breaks down a new one should be bought.
\subsubsection{Infinite Wheelchair maintainers}
If there are infinite wheelchair maintainers, all wheelchairs that break down can and will be fixed in time and thus no wheelchair will bought new. This does make it really expensive for KLM because they have to pay for all the salaries.
\subsubsection{Productive working}
If the wheelchair maintainers are really productive they can fix 48 wheelchairs a day. This means less people will have to be employed.
\subsubsection{Not productive working}
If someone is not working productively, that means that he does not fix any wheelchairs. So you do not actually need them. it is a waste of money
\subsection{Transfer}
Transfer has a few formulas with it, for example distance between the gates, amount of disabled people who need to be transferred and the time between to flights. \\
So what happens when you find the limits of those, the most distant gate or the same gate, no disabled people or a plane full of disabled people and almost no transfer time or infinite or really long transfer time.\\
\begin{equation}
Totaldistance = 2 * distance *Nrdisabled
\end{equation}
\begin{equation}
Totaldistance1Escort= Vescort * TimeBetweenFlights
\end{equation}
\begin{equation}
\lim_{distance \rightarrow \infty} TotalDistance = 2 * distance * NrDisabled 
\end{equation}
\begin{equation}
 = \lim_{NrDisabled \rightarrow \infty} TotalDistance = 2 * distance * NrDisabled = \infty
\end{equation}
\begin{equation}
\lim_{distance \rightarrow 0} TotalDistance = 2 * distance * NrDisabled 
\end{equation}
\begin{equation}
= \lim_{NrDisabled \rightarrow 0} TotalDistance = 2 * distance * NrDisabled = 0
\end{equation}
\begin{equation}
\lim_{TimeBetweenDistance \rightarrow \infty} TotalDistance1Escort = Vescort*TimeBetweenFlights 
\end{equation}
\begin{equation}
= \lim_{Vescort \rightarrow \infty} TotalDistance1Escort = Vescort*TimeBetweenEscort = \infty
\end{equation}
\begin{equation}
\lim_{TimeBetweenDistance \rightarrow 0} TotalDistance1Escort = Vescort*TimeBetweenFlights 
\end{equation}
\begin{equation}
= \lim_{Vescort \rightarrow 0} TotalDistance1Escort = Vescort*TimeBetweenEscort = 0
\end{equation}

\subsubsection{Same gate}
The nearest gate to drop of someone is the gate he arrives at. This means the distance is zero so no distance, so no escort is needed to push him to his gate. Even if there are a hundred disabled people, even if it is a plane full of disabled people and you have 10 transfer seconds, you still do not need any escorts, because they already are at the place they need to be.
\subsubsection{farthest gate}
The gate which is most distant. If you only need to take one person there, you also only need one person to push him there. If there are multiple people who need to go to the most distant gate, you will need more people. It is only possible to push one person at a time for a escort. This means that if the plane is full of disabled people, you will need 550 escorts to push all those people to the same place. If the time in which this has to be done is really small and actually not realistic, only then it is allowed to prioritize other planes and let some people wait. There is no other reason why people should have to wait.
\subsubsection{No disabled people}
If there are no disabled people, no escorts are needed, so no extra money is involved in a flight.
\subsubsection{Plane full of disabled people}
A plane full of disabled people contains approximately 550 persons. All those persons need to be brought to the right gate. There are no mistakes allowed, else multiple planes might have a delay. So 550 escorts are needed to bring all those people to different or the same gate. If those people all need to go to different planes and some of those planes have a real high transfer time, then it is possible to use less escorts. So an escort could first take the people with an higher priority, which means that their plane leaves earlier. After they have brought the first person to their plane, they go back to the previous gate and pick up another person.
\subsection{wheelchairs}
Wheelchairs have two functions, one for the price and one for the devaluing of the wheelchair. What if a wheelchair is free or really expensive and what if they do not decrement or they are immediately worthless?\\
\begin{equation}
CostWheelchairs = Price * AmountOfWheelchairs
\end{equation}
\begin{equation}
CostWheelchairsPerSecond = \frac{TotalCostWheelchair}{TimeTillDestruction}
\end{equation}
\begin{equation}
CostDecrease = CostWheelchairPerSecond * TimeBetweenFlights * NrEscorts
\end{equation}
\begin{equation}
CostMaintenance = \frac{CostDecrease}{2}
\end{equation}
\subsubsection{Free}
If the wheelchairs are free, no wheelchair maintainers are needed. So wheelchairs will actually cost nothing.
\subsubsection{No decrement}
If the value of a wheelchair does not decrement, they can be sold for the same price as you have bought them. This also means that wheelchairs will never be broken too much. So they will live forever.
\subsubsection{immediately valueless}
If wheelchairs are immediately valueless after you have bought them, it is best to check the duration of a wheelchair and decide after that wether or not to buy a new wheelchair or fix the wheelchair.

\begin{equation}
\end{equation}



\end{document} 